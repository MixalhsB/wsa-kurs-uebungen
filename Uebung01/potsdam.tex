\documentclass[a4paper]{article}
\usepackage[utf8]{inputenc}
\usepackage[T1]{fontenc}
\usepackage{lmodern}
\usepackage[ngerman]{babel}

\title{Wissenswertes über Potsdam}
\date{}

\begin{document}

\maketitle

Potsdam, die Hauptstadt des Bundeslandes Brandenburg, hat eine reiche Geschichte, die mehrere Jahrhunderte zurückreicht. Während des 18. und 19. Jahrhunderts diente die Stadt als Residenzstadt der preußischen Könige. Potsdam war auch Schauplatz der Potsdamer Konferenz im Jahr 1945, bei der nach dem Ende des Zweiten Weltkriegs entscheidende Beschlüsse über die zukünftige Gestaltung Deutschlands und Europas gefällt wurden.

Potsdam ist bekannt für seine historischen Schlösser und Gärten, die zum UNESCO-Weltkulturerbe gehören.

Schloss Sanssouci, oft als das \glqq deutsche Versailles\grqq\ bezeichnet, ist eines der bekanntesten Hohenzollernschlösser in Deutschland. Friedrich der Große ließ es im 18. Jahrhundert als Sommerpalast errichten.

Der Filmpark Babelsberg gibt einen Einblick in die Welt des Films und Fernsehens. Hier können Besucher Kulissen, Requisiten und Studio-Sets von bekannten Filmproduktionen betrachten und mehr über die Geschichte des Filmstudios erfahren.

Die Universität Potsdam wurde 1991 gegründet und hat sich seitdem zu einer bedeutenden Bildungs- und Forschungseinrichtung in Brandenburg entwickelt. Mit ihren zahlreichen Fachbereichen, Forschungsinstituten und internationalen Kooperationen bietet sie Studierenden und Forschenden vielfältige Möglichkeiten.

\end{document}

